% !TeX root = ../main.tex

\begin{acknowledgements}

在研究学习期间,我有幸得到陈晓辉老师,尹华锐老师和姜文彬学长的指导与帮助。

依稀记得,当初我因考研导致论文开题延期。在教务系统上,大概6个人选择了陈老师的这个课题,但陈老师最后选择相信我。实验开始时,
我对神经网络完全是零基础,面对ffplay的3000多行源码更是感到力不从心,多亏了姜文彬学长的帮助,才理解ffplay的工作流程,并
开始入门pytorch。实验进行到一半,我错误的将显著性检测理解为物体检测,还好陈老师及时提醒和指正,从而让我避免在错误的方向上
花更多时间,除此之外,陈老师还细心教导我们论文引用部分的写作方法。到了实验后期,我成功将显著性检测模型嵌入到ffplay中,
但是视频帧率比较低,不如人意,陈老师这时又热心指导我尝试使用GPU加速神经网络和DMA方式转移数据,并耐心告诫我们论文写作时的
注意事项。就这样,一步一步,从无到有,我顺利地完成了毕业设计,此时的心情是欣慰和感激。

“好雨知时节,当春乃发生;随风潜入夜,润物细无声”,非常感谢陈晓辉老师,尹华锐老师和姜文彬学长的细心栽培和无私帮助。祝愿陈晓辉老师
和尹华锐老师日后能培养出更多杰出学子,桃李满天下;也祝研二的姜文彬学长能早早发出优秀的paper,百尺竿头,更进一步!

\end{acknowledgements}

% !TeX root = ../main.tex

\ustcsetup{
  keywords = {
    显著性检测, ffmpeg, poolnet, libtorch
  },
  keywords* = {
    saliency detection, ffmpeg, poolnet, libtorch
  },
}

\begin{abstract}

FFmpeg作为先进的跨平台的开源视频处理软件,正在越来越多地被使用。但其缺少对显著性检测的支持,难以实现与显著性检测等常用图像处理算法的结合。
随着 GPU 的飞速发展,大量优秀的基于神经网络的图像处理算法也纷纷涌现。如果能够实现这些算法与FFmpeg的结合,可以充分拓展FFmpeg的应用场景。
本文旨在探究在FFmpeg中嵌入神经网络算法的可行途径,并嵌入显著性检测算法。但 FFmpeg 源码工程量巨大,每次测试编译,时间久,效率低。
本文提出了一种巧妙的办法,将神经网络算法嵌入FFmpeg的播放工具ffplay中并单独编译。
实验结果显示可以正常运行,速度可达12fps,并且此方法通用性较好,亦可适用于其他基于神经网络的图像处理算法。
工程代码地址在 \url{https://github.com/hanjialeOK/YOLOv3-in-FFmpeg/tree/saliency}

\end{abstract}

\begin{abstract*}

As an advanced cross-platform and open-source video processing software, FFmpeg is being used more and more widely. 
However, it lacks the support of saliency detection, so it is difficult to combine with saliency detection and other common image processing algorithms. 
With the rapid development of GPU, a large number of excellent image processing algorithms based on neural network have emerged. 
If these algorithms can be combined with FFmpeg, the application of FFmpeg can be fully expanded. 
This paper aims to explore the feasible way of putting neural network algorithm in FFmpeg, such as the saliency detection. 
But the source code of FFmpeg is a huge amount of work, each test compilation would take a long time with low efficiency. 
This paper presents an ingenious method, which embeds the neural network algorithm into ffplay and compiles it separately. 
The experimental result show that the system can operate normally, and the speed can reach 12 fps. 
In addition, This method has good generality and can also be applied to other image processing algorithms based on neural network.
Code can be found at \url{https://github.com/hanjialeOK/YOLOv3-in-FFmpeg/tree/saliency}.

\end{abstract*}